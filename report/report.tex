\documentclass{article}
\usepackage{amsfonts}
\usepackage{amssymb}
\usepackage{amsmath}
\usepackage{graphicx}
\usepackage{mathtools}
\usepackage{float}
\usepackage{bm}
\usepackage[margin=1in]{geometry}
\newcommand\bd{\bullet}
\newcommand\wt{\circ}
\makeatletter
% we use \prefix@<level> only if it is defined
\renewcommand{\@seccntformat}[1]{%
  \ifcsname prefix@#1\endcsname
    \csname prefix@#1\endcsname
  \else
    \csname the#1\endcsname\quad
  \fi}
% define \prefix@section
\newcommand\prefix@section[1]{\thesection. #1}
\newcommand\userule[1]{\mathrel{\overset{\makebox[0pt]{\mbox{\normalfont\tiny\sffamily #1}}}{=}}}
\makeatother
\title{BDL Coursework}
\author{s1554741}
\date{\today}

\begin{document}
    \maketitle

    \section{The King of Ether}
    The highest level overview of the smart contract for the
    King of Ether is that anyone can become the King by
    paying sufficient amount of Ether (at least as much as
    the last King).
    \\ \\
    When the contract is created, we are saving the
    information about the address of the owner in the
    \verb|owner| variable and then automatically the creator
    becomes the King with the message \verb|Let's play a game...| 
    and with the value 1 wei.
    \\ \\
    Then we have multiple functions that allow other parties
    to take part in the game and become the King themselves.
    \verb|claimThrone(string message)| function allows the
    user to pay a certain amount of Ether (\verb|payable|
    keyword) and if the value is greater or equal from the
    highest value (value that the last King payed), then the
    user becomes the King with the message they passed to
    the function. There is also a restriction on the amount
    of Ether one can use to become the King (50 Ether) that
    can be lifted by the owner of the contract as described
    later. Moreover, any time a new King is determined, the
    last King's earnings are saved in the \verb|earnings|
    map under his address. This way any user that was the
    King but got dethroned can recover their money using the
    \verb|withdraw()| function. 
    \\ \\
    We also have one getter in the form of
    \verb|getKingsTotal()| that allows to see how many kings
    were there in total over the lifespan of the contract
    and \verb|raiseRestriction()| that allows the owner to
    lift the restriction described two paragraphs above and
    allow users to pay more than 50 Ether to become the
    King.
    \\ \\
    Thankfully I managed to become the King at one point
    too. The ID of the transaction was 
    \[
        \texttt{0x903d0a4f95656cbfdbd520898fdfe811d39119e0d229a4effeb14a93d353a79f},
    \]
    my address is
    \[
        \texttt{0x47ADEE763A7BDE2a03c029725C5f7c9315f3B42a}
    \]
    and the message I used for the transaction was
    "\verb|Test transaction please ignore|".

    \section{Rock-paper-scissors}

    \subsection{High level overview} \label{s:overview}
    The most general idea behind my implementation is that
    both players have to pay the same amount of Ether (say
    \(2s\)) and then after the game is ended one of them can
    win at most \(s\) from the other. This design decision
    was made to incentivise both players to finish the game
    even in case of a loss.
    \newline

    \noindent The game starts by a user calling the
    \verb|play| function with a hashed value of of their
    choice along with a nonce (for more information about
    the commitment scheme refer to the section
    \ref{s:scheme}). Then another person can join the game
    by calling the \verb|play| function again with the same
    amount of Ether as the first player and their own hashed
    nonce-choice pair.
    \newline

    \noindent After both players have joined and made their
    commitments, the game moves to the so-called `reveal
    phase', where both players (in any order) have to, as
    the name suggests, reveal their choices by calling the
    \verb|reveal| function and providing their choice and
    the nonce. After both players have done so, the game
    moves on to the final `claim prize phase'.
    \newline

    \noindent In the last phase the players can, in any
    order, claim their Ether back. Notice that, as mentioned
    a few paragraphs above, {\itshape both} players have to
    actually call the \verb|claim| function because they
    both always get some money back - following from our
    earlier example with \(2s\), in case of a win the player
    finishes the game with \(3s\), in case of a tie with
    \(2s\) and in case of a lose with \(1s\). After both
    players have claimed their Ether the game is reset and a
    new round can be started.

    \subsection{Game termination}
    The scenario described above refers to the expected game
    flow, however it may be the case that either one or both
    of the players stop the game mid-through. For this
    eventuality, there is a timer that will allow a player
    to reset the game to the start state if any of the
    players is inactive for a set amount of time (2
    minututes).
    \newline

    \noindent The way the timer works is that if during the
    game any of the players is inactive for more than 2
    minutes, then any player is able to reset the game to
    the state where a new game can be started. If both
    players fail to reveal their values, then any user on
    the blockchain can call the \verb|claim| function that
    will send the stakes back to both players from the last
    game and allow a new game to be initiated. On the other
    hand, if just one of the players reveals their value and
    the second one doesn't, the person who doesn't reveal
    their value gets penalised by not getting any of the
    Ether back.
    \newline

    \noindent It is also worth pointing out that in the
    current version of the game, once the game is finished
    and the result is determined, both players are expected
    to claim back their Ether, because otherwise a new game
    won't be able to start. This is made with the assumption
    that players are reasonable and will always want their
    Ether back.

    \subsection{Commitment scheme details} \label{s:scheme}
    The commitment scheme used in the implementation is
    written with the intention of maximum security and equal
    gas costs for both players during a round. The protocol
    starts by both players picking a random nonce and a
    choice (rock/paper/scissors as either 1, 2 or 3
    respectively), hashing both values using Solidity SHA3
    function and then sending the hashes to the smart
    contract. Once both players have done that, they reveal
    what value they commited to by providing both the nonce
    and the choice they made. Notice that this works due to
    the one-way and collision resistance of hash functions -
    the player cannot reverse hash function to obtain the
    other players choice but also none of the players is
    able to change their choice because that would require
    finding a different nonce that would produce the exact
    same hash.
    \newline

    \noindent Moreover, this seems like a good choice for
    this protocol due to the fact that both players have to
    perform the same exact operations which makes the gas
    costs for both players similar in the first two phases
    of the protocol (`start' and `reveal' phases).
    \newline

    \noindent In terms of nonce generation, a player could
    technically come up with a nonce themself, however that
    may be ill-advised due to low ability to create actual
    randomness by people. Moreover, asking players to create
    the nonces by themselves would most likely result in
    short nonces that would be easy to find. Therefore along
    with the contract code, I have created a short Python
    script that, given user's choice (1, 2 or 3) generates a
    nonce (by taking a random value between \(2^{128}\) and
    \(2^{256}\)) and then hashes the nonce along with the
    user's choice. One could argue that Python's
    \verb|randint| function is not safe enough, although it
    creates enough randomness to ensure that none of the
    players will be able to crack the other player's nonce
    in the time period that the game is on.

    \subsection{Gas cost analysis}
    Gas is an intrinsic part of the Ethereum system that
    allows for the existence of smart contracts. In
    contracts that consider only a case where one user will
    interact with the contract we usually only care to make
    the gas costs as small as possible. However in the case
    where the contract is fundamentally implemented to allow
    interaction between two parties, we not only have to
    care about making the gas costs as low as possible, but
    also to make them as {\itshape equal} as possible.
    Unfortunately due to the nature of gas it is physically
    impossible to make the contract use the same exact
    amount of gas for both parties. We can however try to
    make the difference as small as possible.
    \newline

    \noindent As mentioned in the \ref{s:overview} section,
    the idea behind the game was to make it as symmetric as
    possible so that both players have to make almost
    exactly the same steps during the execution of one round
    of rock-paper-scissors game. With the current
    implementation of the contract, the player that
    initiates the game has to pay around 13\% more gas than
    the player joining second over the course of one round
    of the game.
    \newline

    \noindent The difference in gas costs is not extremely
    big, however such an implementation is obviously far
    from perfect. First off, if parties know that the player
    initiating the game will have to pay more gas in order
    to finish the game, noone may want to be the person
    starting the game, so we could potentially consider a
    slight change to the protocol that would result in a
    lower cost for the {\itshape first} player instead of
    the second to actually incentivise players to start
    games. Secondly, since the implementation tries to make
    gas costs similar for both players, there is potentially
    a lot of space in the code for optimisation that could
    be done (and would result in less overall gas costs),
    however that could mean bigger discrepancy in the gas
    costs between the players. There is definitely a
    trade-off here that can be best resolved by trial and
    error with different approaches.
    
    \subsection{Attempts at improving the gas costs}
    Gas system in Solidity is surprisingly (but
    understandably) complex. Every `machine code' operation
    has its associated gas cost that the party calling the
    function has to pay when said operation is executed.
    While analysing the intricacies of Solidity gas costs, I
    have come up with a few ideas of how the code could
    possibly be adjusted in order to equalise/minimise gas
    usage.

    \subsubsection{Storage and non-zero states}
    First very interesting thing about gas costs is the fact
    that changing a value from a so-called `zero state' to a
    `non-zero state' results in a SSTORE operation that
    costs 20000 gas \cite{ypaper}. In my contract this
    resulted the first game after deployment being
    significantly (\(\sim\)25\%) more expensive than any
    subsequent game. However, a few small tweaks (such as
    changing from \verb|true| and \verb|false| to \verb|1|
    and \verb|2| in the constructor) resulted in a code that
    had no `zero states' at the beginning of the contract
    and consequently removed all the SSTORE operations. This
    seems quite unintuitive, but as far as my understanding
    goes, working with non-zero integers is more gas
    efficient than working with boolean values in Solidity.
    \newline

    \noindent Another interesting fact about the gas costs
    is that even if an operation doesn't change from a `zero
    state' of a variable to a `non-zero state', it still
    costs 5000 gas \cite{ypaper}. Therefore one idea for
    reducing the amount of gas required for a game would be
    to get rid of some not necessarily required global
    variables in the code. For example, we could potentially
    get rid of the \verb|has_revealed| variable of each
    player and try to work with just their \verb|choice| -
    set choice to a value not in \(\{1,2,3\}\) when they
    haven't yet revealed their choice. This way we would
    save up on all the transactions that include changing
    the state of the \verb|has_revealed| variable currently.
    \newline

    \noindent One more solution that may result in lower gas
    costs is shortening the game. In the current
    implementation both players have to call the contract
    three times during the execution of a single game. We
    could, at the cost of gas assymmetry make the game
    shorter and include the Ether claiming phase of the game
    in the reveal phase and use \verb|claim| only if one of
    the players doesn't finish the game in time. A bit of
    testing shows that this approach indeed results in
    smaller gas costs in comparison to the original method
    (\(\sim\)200000 for player 1 and \(\sim\)150000 for
    player 2), however the gas costs differ by around 25\%
    between the players.
    \newline

    \noindent The lesson here should be that storage in
    Solidity is considered very expensive and the more
    `stateless' the contract the cheaper it will be.

    \subsubsection{Asymmetric protocol}
    Another approach that I have considered for the contract
    is an asymmetric protocol where the first player starts
    the game, the second one joins and then the one who
    reveals second gets their money during the reveal call,
    whereas the other player has to call the \verb|claim|
    method in order to retrieve the ether. Surprisingly,
    even though the asymmetric nature could lead to an
    assumption that gas costs cannot be similar, this
    protocol seems to perform best in this regard. Both
    players pay roughly 220000 gas for the whole game which
    is at the same level as the original game but with
    around 5\% difference between the gas costs of each
    player. This seems like a good approach although again,
    thorough testing should be done in order to tell with
    hundred percent certainty that this is the way to go.

    \section{Game code}
    Below you can find the game code as well as the code for
    the nonce and hash generation using Python.


\end{document}
