\documentclass{article}
\usepackage{amsfonts}
\usepackage{amssymb}
\usepackage{amsmath}
\usepackage{graphicx}
\usepackage{mathtools}
\usepackage{makecell}
\usepackage{float}
\usepackage{bm}
\usepackage[hidelinks]{hyperref}
\usepackage[margin=1in]{geometry}
% Copyright 2017 Sergei Tikhomirov, MIT License
% https://github.com/s-tikhomirov/solidity-latex-highlighting/

\usepackage{listings, xcolor}

\definecolor{verylightgray}{rgb}{.97,.97,.97}

\lstdefinelanguage{Solidity}{
	keywords=[1]{anonymous, assembly, assert, balance, break, call, callcode, case, catch, class, constant, continue, constructor, contract, debugger, default, delegatecall, delete, do, else, emit, event, experimental, export, external, false, finally, for, function, gas, if, implements, import, in, indexed, instanceof, interface, internal, is, length, library, log0, log1, log2, log3, log4, memory, modifier, new, payable, pragma, private, protected, public, pure, push, require, return, returns, revert, selfdestruct, send, solidity, storage, struct, suicide, super, switch, then, this, throw, transfer, true, try, typeof, using, value, view, while, with, addmod, ecrecover, keccak256, mulmod, ripemd160, sha256, sha3}, % generic keywords including crypto operations
	keywordstyle=[1]\color{blue}\bfseries,
	keywords=[2]{address, bool, byte, bytes, bytes1, bytes2, bytes3, bytes4, bytes5, bytes6, bytes7, bytes8, bytes9, bytes10, bytes11, bytes12, bytes13, bytes14, bytes15, bytes16, bytes17, bytes18, bytes19, bytes20, bytes21, bytes22, bytes23, bytes24, bytes25, bytes26, bytes27, bytes28, bytes29, bytes30, bytes31, bytes32, enum, int, int8, int16, int24, int32, int40, int48, int56, int64, int72, int80, int88, int96, int104, int112, int120, int128, int136, int144, int152, int160, int168, int176, int184, int192, int200, int208, int216, int224, int232, int240, int248, int256, mapping, string, uint, uint8, uint16, uint24, uint32, uint40, uint48, uint56, uint64, uint72, uint80, uint88, uint96, uint104, uint112, uint120, uint128, uint136, uint144, uint152, uint160, uint168, uint176, uint184, uint192, uint200, uint208, uint216, uint224, uint232, uint240, uint248, uint256, var, void, ether, finney, szabo, wei, days, hours, minutes, seconds, weeks, years},	% types; money and time units
	keywordstyle=[2]\color{teal}\bfseries,
	keywords=[3]{block, blockhash, coinbase, difficulty, gaslimit, number, timestamp, msg, data, gas, sender, sig, value, now, tx, gasprice, origin},	% environment variables
	keywordstyle=[3]\color{violet}\bfseries,
	identifierstyle=\color{black},
	sensitive=false,
	comment=[l]{//},
	morecomment=[s]{/*}{*/},
	commentstyle=\color{gray}\ttfamily,
	stringstyle=\color{red}\ttfamily,
	morestring=[b]',
	morestring=[b]"
}

\lstset{
	language=Solidity,
	backgroundcolor=\color{verylightgray},
	extendedchars=true,
	basicstyle=\footnotesize\ttfamily,
	showstringspaces=false,
	showspaces=false,
	numbers=left,
	numberstyle=\footnotesize,
	numbersep=9pt,
	tabsize=2,
	breaklines=true,
	showtabs=false,
	captionpos=b
}
\newcommand\bd{\bullet}
\newcommand\wt{\circ}
\makeatletter
% we use \prefix@<level> only if it is defined
\renewcommand{\@seccntformat}[1]{%
  \ifcsname prefix@#1\endcsname
    \csname prefix@#1\endcsname
  \else
    \csname the#1\endcsname\quad
  \fi}
% define \prefix@section
\newcommand\prefix@section[1]{\thesection. #1}
\newcommand\userule[1]{\mathrel{\overset{\makebox[0pt]{\mbox{\normalfont\tiny\sffamily #1}}}{=}}}
\makeatother
\title{BDL Coursework}
\author{Mateusz Parafinski\\\small s1554741}
\date{\today}

\begin{document}
    \maketitle

    \section{The King of Ether}
    The highest level overview of the smart contract for the
    King of Ether is that anyone can become the King by
    paying sufficient amount of Ether (at least as much as
    the last King).
    \newline

    \noindent When the contract is created, we are saving
    the information about the address of the owner in the
    \verb|owner| variable and then automatically the creator
    becomes the King with the message \verb|Let's play a game...| 
    and with the value 1 wei.
    \newline

    \noindent Then we have multiple functions that allow
    other parties to take part in the game and become the
    King themselves. \verb|claimThrone(string message)|
    function allows the user to pay a certain amount of
    Ether (\verb|payable| keyword) and if the value is
    greater or equal to the highest value (value that the
    last King payed), then the user becomes the King with
    the message they passed to the function. There is also a
    restriction on the amount of Ether one can use to become
    the King (50 Ether) that can be lifted by the owner of
    the contract. Moreover, any time a new King is
    determined, the last King's earnings are saved in the
    \verb|earnings| map under his address. This way any user
    that was the King but got dethroned can recover their
    money with the bonus equal to the difference between
    their value and the new King's value using the
    \verb|withdraw()| function.
    \newline

    \noindent We also have one getter in the form of
    \verb|getKingsTotal()| that allows to see how many kings
    were there in total over the lifespan of the contract.
    \newline

    \noindent Thankfully I managed to become the King at one
    point too. The ID of the transaction was 
    \[
        \texttt{0x903d0a4f95656cbfdbd520898fdfe811d39119e0d229a4effeb14a93d353a79f},
    \]
    my address is
    \[
        \texttt{0x47ADEE763A7BDE2a03c029725C5f7c9315f3B42a}
    \]
    and the message I used for the transaction was
    ``\verb|Test transaction please ignore|''.

    \section{Rock-paper-scissors}

    \subsection{High level overview} \label{s:overview}
    The most general idea behind my implementation is that
    both players have to pay the same amount of Ether (say
    \(x\)) and then after the game is ended one of them can
    win at most \(\frac{x}{2}\) from the other (with the
    rest being the desposit returned to both players). The
    design decision of having a big deposit was made to
    incentivise both players to finish the game even in case
    of a loss.
    \newline

    \noindent The game starts by a user calling the
    \verb|play| function with a hashed value of their choice
    along with a nonce (for more information about the
    commitment scheme refer to Section \ref{s:scheme}). Then
    another person\footnote{The second player could actually
    be the same person as the first one for security reasons
    - see Section \ref{s:playown}.} can join the game by
    calling the \verb|play| function again with the same
    amount of Ether as the first player and their own hashed
    nonce-choice pair.
    \newline

    \noindent After both players have joined and made their
    commitments, the game moves to the so-called `reveal
    phase', where both players (in any order) have to, as
    the name suggests, reveal their choices by calling the
    \verb|reveal| function and providing their choice and
    the nonce. After both players have done so, the game
    moves on to the final `claim prize phase'.
    \newline

    \noindent In the last phase the players can, in any
    order, claim their Ether back. Notice that, as mentioned
    a few paragraphs above, {\itshape both} players have to
    actually call the \verb|claim| function because they
    both always get some money back - following from our
    earlier example with stake \(x\), in case of a win the
    player finishes the game with \(\frac{3x}{2}\), in case
    of a tie with \(x\) and in case of a lose with
    \(\frac{x}{2}\). After both players have claimed their
    Ether, the game is reset and a new round can be started.

    \subsection{Game termination}
    The scenario described above refers to the expected game
    flow, however it may be the case that either one or both
    of the players stop the game mid-through. For this
    eventuality, there is a timer that will allow anyone to
    reset the game to the start state in case of inactivity.
    \newline

    \noindent The timer has a threshold of 2 minutes and is
    updated to current time at every activity after the game
    starts (i.e. after both players provide a commitment).
    Now, if the timer hasn't been updated in 2 minutes, then
    any blockchain user can interact with the contract by
    calling the \verb|claim| function. The function will,
    based on whether any player has revealed their value or
    not, send either half of the money to both players (so
    each of them will get their investment back) or
    everything to just one of them in case of just one
    player failing to reveal - note that we are penalising
    inactivity more than actual loss in the game, the player
    doesn't get anything back if they bail.
    \newline

    \noindent It is also worth pointing out that in the
    current version of the game, once the game is finished
    and the result is determined, both players are expected
    to claim back their Ether, because otherwise a new game
    won't be able to start. This is made with the assumption
    that players are reasonable and will always want their
    Ether back and that withholding the funds inside the
    contract has no use to any party.

    \subsection{Commitment scheme details} \label{s:scheme}
    The commitment scheme used in the implementation is
    written with the intention of maximum security and equal
    gas costs for both players during a round. The protocol
    starts by both players picking a random nonce and a
    choice (rock/paper/scissors as either 0, 1 or 2
    respectively), hashing both values using Solidity SHA3
    function and then sending the hashes to the smart
    contract. Once both players have done that, they reveal
    what value they commited to by providing both the nonce
    and the choice they made. Notice that this protocol
    works due to the fact that hash functions are one-way
    and collision resistant - the player cannot reverse hash
    function to obtain the other players choice but also
    none of the players is able to change their choice
    because that would require finding a different nonce
    that would produce the exact same hash.
    \newline

    \noindent Moreover, this seems like a good choice for
    this protocol due to the fact that both players have to
    perform the same exact operations which makes the gas
    costs for both players similar in the first two phases
    of the protocol (`start' and `reveal' phases) - see
    Section \ref{s:gascost}.
    \newline

    \noindent In terms of nonce generation, a player could
    technically come up with a nonce themselves, however
    that may be ill-advised due to the fact that people are
    bad at making things random. Moreover, asking players to
    create the nonces by themselves would most likely result
    in short nonces that would be easy to find. Therefore
    along with the contract code, I have created a short
    Python script that, given user's choice (0, 1 or 2)
    generates a nonce (by taking a random value between
    \(2^{128}\) and \(2^{256}\)) and then hashes the nonce
    along with the user's choice. One could argue that
    potentially Python's \verb|randint| function may not be
    safe enough, although it creates enough randomness to
    ensure that none of the players will be able to crack
    the other player's nonce in the time period that the
    game is on.

    \subsection{Transaction history of the game execution}
    Now that we know how the game works and what steps are
    needed to play one game of rock-paper-scissors on the
    contract, we can go through a history of transactions
    that were made to have a go at it. The game was played
    on my contract with the address
    \[
        \texttt{0x991a0c2c51d9b5772151fbd92569777869cca499}.
    \]

    \noindent The game started by me calling the function
    \verb|play| with my hash as the parameter:
    \[
        \text{hash 1: }\texttt{57422681862702588708287488993004352428084682103926404673879216601876390030369}
    \]

    \noindent The transaction ID for this call was 
    \[
        \texttt{0x260d376fe852f28e8ccbd3aaec29fcdc1f50a43bc0bc762d1889d5a7bfdb2af4}.
    \]

    \noindent Next, my friend joined the game by also
    providing their hash to the \verb|play| function:
    \[
        \text{hash 2: }\texttt{91798907740737038578965130316324719596937870764908358325218676231228180405586}.
    \]

    \noindent The transaction ID for this call was
    \[
        \texttt{0x970781723b179228e55c48a26aa2955e8e81226ded0f9a0d211aa2e498e2fec4}.
    \]

    \noindent Now, the contract was waiting for us to reveal
    the values in any order. I was the first one to reveal
    mine by calling the \verb|reveal| function with the
    following parameters:
    \[
    \begin{split}
        &\text{choice 1: }\texttt{1 (Paper)}\\
        &\text{nonce 1: }\texttt{68467566110241134270540132358642702736493862980289395847769434602692193680791}.
    \end{split}
    \]

    \noindent The transaction ID for this call was
    \[
        \texttt{0x75e54a594aa17d23a8bf03e9bd1effe876de571bc2df2a918accc73071b88092}.
    \]

    \noindent Subsequently, my friend had to reveal their
    choice too using the \verb|reveal| function:
    \[
    \begin{split}
        &\text{choice 2: }\texttt{1 (Paper)}\\
        &\text{nonce 2: }\texttt{65958972353641206791692374757714317624675152672570927745663838423713800289627}.
    \end{split}
    \]

    \noindent The transaction ID for this call was
    \[
        \texttt{0xc959192c57ecb834f31507923ff363361ad7b4f3d91afeb7f06349d04f4a2081}.
    \]

    \noindent At this point the game was technically
    finished and the winner (or actually lack thereof) was
    determined, but we still had to claim back our money
    with the help of \verb|claim| function. I was the first
    one to do that. I claimed back my 1 Ether and the
    transaction ID for this was
    \[
        \texttt{0x83c223afff33453d08ae31aaab2db91f274b8bd0f2a83f150662b5a946d540ce}.
    \]

    \noindent The last thing left to do was for my friend to
    get their money back. He called the \verb|claim|
    function again and retrieved his 1 Ether. The transaction
    ID for this call was
    \[
        \texttt{0xcfa840a8a7ce927f95db7209fee0d6215f7f0e500b5d00f07d9332008fbc40aa}.
    \]

    \noindent At this point the contract is ready for the
    next two players to join the game as can be checked by
    calling the \verb|getGamePhase| function.

    \subsection{Gas cost analysis} \label{s:gascost}
    Gas is an intrinsic part of the Ethereum system that
    allows for the existence of smart contracts. In
    contracts which don't allow interaction between multiple
    users, we usually only care to make the gas costs as
    small as possible. However in the case where the
    contract is fundamentally implemented to allow
    interaction between two parties, we not only have to
    care about making the gas costs as low as possible, but
    also to make them as {\itshape equal} as possible.
    Unfortunately due to the nature of gas it is physically
    impossible to make the contract use the same exact
    amount of gas for both parties. We can however try to
    make the difference as small as possible.
    \newline

    \noindent As mentioned in Section \ref{s:overview}, the
    idea behind the game was to make it symmetric so that
    both players have to take almost exactly the same steps
    during the execution of one rock-paper-scissors game.
    With the current implementation of the contract, the
    player that initiates the game has to pay around 10\%
    more gas than the player joining second over the course
    of one the game. Table \ref{tab:gasTable} represents the
    average gas costs for each order of the play possible:
    \newline

    \begin{table}[H]
        \centering
        \begin{tabular}{|l l l l l|}
            \hline
            Player & Start phase [gas] & Reveal phase [gas]
            & Claim phase [gas] & Average cost [gas] \\
            \hline
            \multicolumn{4}{c}{} \\[-.5pt]
            \multicolumn{5}{c}{{\bfseries Player 1 starts, Player 1
            reveals 1\textsuperscript{st}, Player 1 claims
            1\textsuperscript{st}}} \\
            \hline
            1 & 50008 & 54393 & 36614 & 141014 \\
            2 & 35460 & 60417 & 27096 & 123270 \\
            \hline
            \multicolumn{4}{c}{} \\[-.5pt]
            \multicolumn{5}{c}{{\bfseries Player 1 starts, Player 1
            reveals 1\textsuperscript{st}, Player 2 claims
            1\textsuperscript{st}}} \\
            \hline
            1 & 50008 & 54393 & 26855 & 131255 \\
            2 & 35467 & 60714 & 36570 & 132751 \\
            \hline
            \multicolumn{4}{c}{} \\[-.5pt]
            \multicolumn{5}{c}{{\bfseries Player 1 starts, Player 2
            reveals 1\textsuperscript{st}, Player 1 claims
            1\textsuperscript{st}}} \\
            \hline
            1 & 50008 & 60974 & 36730 & 147712 \\
            2 & 35431 & 54703 & 27299 & 117433 \\
            \hline
            \multicolumn{4}{c}{} \\[-.5pt]
            \multicolumn{5}{c}{{\bfseries Player 1 starts, Player 2
            reveals 1\textsuperscript{st}, Player 2 claims
            1\textsuperscript{st}}} \\
            \hline
            1 & 50008 & 60495 & 26855 & 137358 \\
            2 & 35438 & 54397 & 36570 & 126405 \\
            \hline
        \end{tabular}

        \caption{Average gas costs in different
        configurations. The average was taken by calculating
        all possible combintations of choices for each
        player}
        \label{tab:gasTable}
    \end{table}

    \noindent The difference in gas costs is not extremely
    big, however such an implementation is obviously far
    from perfect. First off, if parties know that the player
    initiating the game will have to pay more gas in order
    to finish the game, noone may want to be the person
    starting the game, so we could potentially consider a
    slight change to the protocol that would result in a
    lower cost for the {\itshape first} player instead of
    the second to actually incentivise players to start
    games. Secondly, since the implementation tries to make
    gas costs similar for both players, there is a lot of
    potential for optimisation in the code (that would
    result in less overall gas costs), however that could
    mean bigger discrepancy in the gas costs between the
    players. There is definitely a trade-off here that can
    be best resolved by trial and error with different
    approaches.
    
    \subsection{Attempts at improving the gas costs}
    Gas system in Solidity is surprisingly (but
    understandably) complex. Every `machine code' operation
    has its associated gas cost that the party calling the
    function has to pay when said operation is executed.
    While analysing the intricacies of Solidity gas costs, I
    have come up with a few ideas of how the code could
    possibly be adjusted in order to equalise/minimise gas
    usage.

    \subsubsection{Storage and non-zero states}
    First very interesting thing about gas costs is the fact
    that changing a value from a so-called `zero state' to a
    `non-zero state' results in a SSTORE operation that
    costs 20000 gas \cite{ypaper}. In my contract this
    resulted in the first game after deployment being
    significantly (\(\sim\)25\%) more expensive than any
    subsequent game. However, a few small tweaks (such as
    changing from \verb|true| and \verb|false| to \verb|1|
    and \verb|2| in the constructor) resulted in a code that
    had no `zero states' at the beginning of the contract
    and consequently removed all the SSTORE operations. This
    seems quite unintuitive, but as far as my understanding
    goes, working with non-zero integers is more gas
    efficient than working with boolean values in
    Solidity\footnote{The gas costs for the SSTORE
    operations were moved to the constructor instead, so the
    owner has to pay more, nothing is free.}.
    \newline

    \noindent Another interesting fact about the gas costs
    is that even if an operation doesn't change from a `zero
    state' of a variable to a `non-zero state' but to a
    different `zero state', it still costs 5000 gas
    \cite{ypaper}. Therefore one idea for reducing the
    amount of gas required for a game would be to get rid of
    some not necessarily required global variables in the
    code. For example, we could potentially get rid of the
    \verb|has_revealed| variable of each player and try to
    work with just their \verb|choice| - set choice to a
    value not in \(\{0,1,2\}\) when they haven't yet
    revealed their choice. This way we would save up on all
    the transactions that include changing the state of the
    \verb|has_revealed| variable.
    \newline

    \noindent One more solution that may result in lower gas
    costs is shortening the game. In the current
    implementation both players have to call the contract
    three times during the execution of a single game. We
    could, at the cost of gas assymmetry make the game
    shorter and include the Ether claiming phase of the game
    in the reveal phase and use \verb|claim| only if one of
    the players doesn't finish the game in time. A bit of
    testing shows that this approach indeed results in
    smaller gas costs in comparison to the original method
    (\(\sim\)120000 for player 1 and \(\sim\)100000 for
    player 2), however gas costs differ by around 20\%
    between the players.
    \newline

    \noindent The lesson here should be that storage in
    Solidity is considered very expensive and the more
    `stateless' the contract the cheaper it will be.

    \subsubsection{Asymmetric protocol}
    Another approach that I have considered for the contract
    is an asymmetric protocol where the first player starts
    the game, the second one joins and then the one who
    reveals second gets their money during the reveal call,
    whereas the other player has to call the \verb|claim|
    method in order to retrieve the ether. Surprisingly,
    even though the asymmetric nature could lead to an
    assumption that gas costs cannot be similar, this
    protocol seems to perform best in this regard. Both
    players pay roughly 120000 gas for the whole game which
    is at the same level as the original game but with
    around 5\% difference between the gas costs of each
    player instead of 10\%. This seems like a good approach
    although again, thorough testing should be done in order
    to tell with hundred percent certainty that this is the
    way to go.

    \subsubsection{Forced order}
    Finally, one last thing to notice when looking at the
    Table \ref{tab:gasTable} is that in the case when player
    1 reveals first and player 2 claims first the gas costs
    are nearly identical. Thus another approach at making
    the costs equal could be dictating a strict order of
    play where player 2 cannot reveal first and player 1
    cannot claim the money first.

    \subsection{Vulnerability analysis}
    When writing a smart contract, many things can go wrong
    or unnoticed and then be used maliciously by other
    parties. I am going to go through some of the
    vulnerabilities I have found when analysing other
    people's code over the past few weeks.

    \subsubsection{Reentrancy}
    One of the very common vulnerabilities of smart
    contracts is the ability to call back a function after
    being transferred some funds before the state of the
    smart contract is updated. Even though most students
    have taken into account the fact that reentrancy is
    dangerous, it is not difficult to forget about some
    possible cases where this may happen. In one example, a
    student made sure to reset the state if the game ended
    in the usual manner. However, in the case where one or
    both of the players were inactive some kind of a reset
    had to be called, the state update was called after
    transfering the money to the players. This obviously
    leads to the possibility of a reentrancy attack. In the
    code, this could look like:

    \begin{lstlisting}[language=Solidity]
/**
...
*/
function claimMoney() {
    if (msg.sender == winner) {
        // here we prevent the reentrancy
        updateState();
        msg.sender.transfer(winnings);
    }
}

function kickInactive() {
    if (now - time > 1 hour) {
        player1.transfer(stake/2);
        player2.transfer(stake/2);
        // here reentrancy is possible
        updateState();
    }
}
/**
...
*/
    \end{lstlisting}

    \noindent Obviously \verb|player1| can exploit the
    vulnerability by simply running the \verb|kickInactive|
    function again when the money is being transferred to
    them and get all the money instead of just half. The
    lesson here is to make sure that every time we are
    either transfering money or calling a function of a
    different contract, the state of our contract doesn't
    allow any unwanted actions to be taken.

    \subsubsection{Checking identities}
    Another relatively simple vulnerability that I have
    noticed is that whenever one of the players gets paid,
    it isn't actually checked {\bfseries who} gets paid and
    thus the same player can call the function in quick
    succession and get all the money out of the contract. An
    example of this attack can be seen here

    \begin{lstlisting}[language=Solidity]
/**
...
*/
function claimWinnings() {
    if (msg.sender == winner) {
        msg.sender.transfer(winnings + deposit);
    } else {
        msg.sender.transfer(deposit);
    }
}
/**
...
*/
    \end{lstlisting}

    \noindent Obviously, this attack only works in a
    symmetric scheme in which both players have to call the
    \verb|claimWinnings| function to claim their money back,
    i.e. if there is some kind of deposit that both of them
    put in at the start of the game. If only the winner gets
    money, then they will get all the money out of the
    contract in the first run of the function anyway.
    Nevertheless, if a contract involves interaction between
    two parties and the way the contract should behave
    differs by player, then we should always check for the
    identity of the player and update the state based on the
    action we have done as in here (based on the code from
    Section \ref{s:code}):
    
    \begin{lstlisting}[language=Solidity]
/**
...
*/
function claimWinnings() {
    player = determinePlayer(msg.sender);
    require(!player.gotPaid, "Player already got paid");
    // here we make sure the state gets updated
    // before the function can be called again
    player.gotPaid = true;

    if (player == winner) {
        player.transfer(winnings + deposit);
    } else {
        player.transfer(deposit);
    }
}
/**
...
*/
    \end{lstlisting}

    \subsubsection{`Inactivity' attack} \label{s:playown}
    This is a bit more subtle vulnerability that makes use
    of the fact that someone has to start the game and wait
    for an opponent. Let's say that happens and we are
    waiting for player 2 to join the game. What player 2
    could do is wait until player 1 forgets that they have
    started the game (or simply gets bored of waiting for an
    opponent) and then start the game and immediately reveal
    their hand.
    \newline

    \noindent Now, we know that to prevent a player from
    never revealing their hand, we need to include a timer
    in the implementation to be able to reset in case of
    inactivity. Therefore, since player 1 has forgot about
    the game/is inactive for some other reason, player 2
    can just wait and claim all the money.
    \newline
    
    \noindent Since most games I have analysed had the same
    model of the game flow, this attack works for most of
    them, but I wouldn't call it an implementation flaw. We
    can prevent this from happening by introducing some
    counter-measures that player 1 can take. One such
    measure would be to enable player 1 to play with
    themselves if noone else decides to play with them,
    which can be seen in the following code:

    \begin{lstlisting}[language=Solidity]
/**
...
*/
function hashHand(bytes32 hash) payable public {
    require(gameState <= 2);
    
    // enforce deposit for not completing protocol punishment
    require(msg.value >= minBuyIn + minDeposit); 
    
    if (gameState == 1) {
        gameState = 2;
        player1State = PlayerState(msg.sender, hash, msg.value - minDeposit);
        // the player stakes exclude the mandatory deposit (it does not take 
        // part in the prize-pool)
        
    } else { // this must be gameState = 2
        gameState = 3;
        player2State = PlayerState(msg.sender, hash, msg.value - minDeposit);
        time = now; // start timer for detecting inactivity
        
        // this is where "residuals", which is the overpaid excess money 
        // that should be returned to their respective owners is tracked
        if (player1State.stake >= player2State.stake) {
            winnings = player2State.stake; // also calculating prize-pool
            p1_residual = player1State.stake - player2State.stake + 1;
            // offseting the residuals by 1 to avoid zeroes
        } else {
            winnings = player1State.stake;
            p2_residual = player2State.stake - player1State.stake + 1;
        }
    }
}
/**
...
*/
function revealHand(uint256 hand) public {
    require(gameState >= 3);
    if (gameState == 3) {

        require(msg.sender == player2State.adress);
        require(sha256(abi.encodePacked(hand)) == player2State.hash);
        // the hash-based commitment scheme check
        gameState = 4;
        
        p2_hand = hand % 10 + 3; // saving with the offset for avoiding zeroes
        time = now; // overwrite timer for detecting inactivity

    } else { // this must be gameState = 4
    
        require(msg.sender == player1State.adress);
        require(sha256(abi.encodePacked(hand)) == player1State.hash);
        
        hand = hand % 10;
        bool isDraw = (hand % 3 == p2_hand % 3);
        bool player2Won = (hand % 3 == (p2_hand + 1) % 3);
        
        if (isDraw){
            endGameGetPaid(player1State.adress, player2State.adress, 
                            player1State.stake, player2State.stake, 0, false);
                            // indicate draw with a 0-value winning,
                            // residuals become the original stakes
        } else {
            if (player2Won) {
                endGameGetPaid(player2State.adress, player1State.adress, 
                                p2_residual, p1_residual, winnings, false);
            } else {
                endGameGetPaid(player1State.adress, player2State.adress, 
                                p1_residual, p2_residual, winnings, false);
            }   
        }
    }
}
/**
...
*/
    \end{lstlisting}

    \noindent We can see that there is nothing here
    preventing player 1, in case of noone else wanting to
    play, to start a game with themselves and play it out
    and get all the money back no matter of the result.
    Another possible approach to solving this issue (with
    possibly smaller gas costs) would be to allow a player
    to reset the game state and get the money back as long
    as there is just one player in the game, e.g. something
    like

    \begin{lstlisting}[language=Solidity]
/**
...
*/
function withdraw() {
    require(gamePhase == WaitingForPlayer2);
    gamePhase = WaitingForPlayers;
    player1.transfer(stake);
}
/**
...
*/
    \end{lstlisting}

    \subsubsection{Hash calculation in the commitment}
    In all the implementations I have looked at, the second
    phase of the game is related to revealing the commited
    value by both players. Usually, the players provide the
    nonce and the choice as the arguments to the revealing
    function which then checks whether the values they have
    given hash to their earlier commitment.
    \newline

    \noindent There are multiple ways that a protocol could
    define `combining' the values of the nonce and the
    choice. I would like to focus here on two that I have
    found to be vulnerable to attacks. From this point on I
    will denote the nonce by \(n\) and the choice by \(c\).
    \newline

    \noindent The first and simpler version combines the
    values of nonce and choice by simply adding them
    together: \(n + c\). One contract that I have looked at
    had this exact procedure for checking the commitment:

    \begin{lstlisting}[language=Solidity]
/**
...
*/
function reveal(uint256 randInt, RPS choice) private {
    require(registered(), "You have not registered for the game!");
    require(gameInProgress(), "At least one player missing!");

    // player1 always reveals his choice and random integer first.
    if (gameState == State.NoReveal1 && msg.sender == player1) {
        // Make sure that the hashes match and thus player1's commitement was valid.

        // *** BELOW IS THE LINE WITH THE VULNERABILITY ***
        require(player1Hash == sha256(bytes32(randInt + uint256(choice))), "Hashes 1 don't match!");

        // Store the random integer and choice of player1.
        player1randInt = randInt;
        player1Choice = choice;
        // Update the game sate to make sure it's player2's turn to reveal next.
        gameState = State.NoReveal2;

        // Update the time stamp to check for timeout later
        timerStart = now;
    }
    /**
    ...
    */
}
/**
...
*/
    \end{lstlisting}

    \noindent As we can see in the code, the contract
    creator decided to combine the nonce and the choice by
    adding them together. However, this leads to a quite
    devastating attack - imagine the player initially chose
    nonce \(n\) and choice \(c\) and got a hash \(h =
    \text{sha256}(n + c)\). Notice that any pair
    \((n-k,c+k)\) will lead to the same hash:
    \[
       \text{sha256}(n-k+c+k) = \text{sha256}(n+c) = h 
    \]

    \noindent which means that essentially a player can
    change their choice at any given point before revealing
    - in particular, they can change it {\itshape after} the
    other person reveals their choice. Obviously this is far
    from desired.
    \newline

    \noindent After I have notified the author of the
    vulnerability, they came up with another protocol that
    instead of adding the values together `appended' the
    choice at the end of the nonce, so basically the joined
    value was calculated as
    \[
        10n + (c\%3)
    \]

    \noindent and then hashed to produce
    \[
        h = \text{sha256}(10n + (c\%3)).
    \]

    \noindent Is this perfectly safe? Certainly the attack
    suggested earlier will not work - providing \(n - k\) as
    nonce and \(c + k\) as choice leads to \(h' =
    \text{sha256}(10n - 10k + ((c + k) \% 3)))\) which with
    very high probability leads to a hash different than our
    original one, \(h \neq h'\). However, is there by any
    chance another attack that could work here? Consider the
    following corrected version of the code from above:

    \begin{lstlisting}[language=Solidity]
function revealRock(uint256 randInt) public{
    // 'revealRock' reveals that the sender choice was Rock, and randInt in the random integer he
    // used to generate the hash.
    reveal(randInt, RPS.Rock); // RPS.Rock is 0
}

function revealPaper(uint256 randInt) public{
    // 'revealPaper' reveals that the sender choice was Paper, and randInt in the random integer he
    // used to generate the hash.
    reveal(randInt, RPS.Paper); // RPS.Paper is 1
}

function revealScissors(uint256 randInt) public{
    // 'revealScissors' reveals that the sender choice was Scissors, and randInt in the random integer he
    // used to generate the hash.
    reveal(randInt, RPS.Scissors); // RPS.Scissors is 2
}

function reveal(uint256 randInt, RPS choice) private {
    require(registered(), "You have not registered for the game!");
    require(gameInProgress(), "At least one player missing!");

    // player1 always reveals his choice and random integer first.
    if (gameState == State.NoReveal1 && msg.sender == player1) {
        // Make sure that the hashes match and thus player1's commitement was valid.
        require(player1Hash == sha256(bytes32(randInt*10 + uint256(choice))), "Hashes 1 don't match!");
        // Store the random integer and choice of player1.
        player1randInt = randInt;
        player1Choice = choice;
        // Update the game sate to make sure it's player2's turn to reveal next.
        gameState = State.NoReveal2;

        // Update the time stamp to check for timeout later
        timerStart = now;
    }
    /**
    ...
    */
}
    \end{lstlisting}

    \noindent I claim that this implementation leads to a
    very effective attack that guarantees \(\frac{2}{3}\)
    chance of a win and \(\frac{1}{3}\) chance of a draw
    against an opponent that chooses each of the three
    possibilities uniformly. The key to the attack is the
    fact that Solidity, similarly to many other programming
    languages, works with numbers using modular arithmetic.
    That is, if we have a variable of type \verb|uint8| with
    value 250, say \verb|uint8 v = 250|, then \verb|v + 10|
    will not result in \verb|v| being equal to 260 but \(260
    \% 256 = 4\). The same goes for \verb|uint256| but the
    modular arithmetic is done modulo \(2^{256}\).
    \newline

    \noindent The idea is that if we manage to overflow the
    value of \verb|randInt*10| in the code above, then maybe
    we are able to find a `collision' that allows us to
    change our choice. Thus we are focused on finding two
    numbers, say \(a\) and \(b\) such that 
    \[
        10a + k = 10b + l \text{ (mod 2}^{256})
    \]

    \noindent where \(k\) and \(l\) are two choices that we could
    potentially make. We can rewrite the equation above to
    produce
    \[
        10(a-b) = l - k \text{ (mod 2}^{256})
    \]

    \noindent so if we set \(x := a - b\), then we are
    looking for an \(x\) and \(m\) such that
    \[
        10x + m2^{256} = l - k.
    \]

    \noindent It's not hard to prove that this has a
    solution if and only if \(\gcd(10, 2^{256}) | l - k\)
    and therefore, since \(\gcd(10, 2^{256}) = 2\) we need
    \(l\) and \(k\) to be such that \(2 | l - k\). Now,
    since the code above only allows \(l,k \in \{0, 1,
    2\}\), this will only work for \(l = 2\) and \(k = 0\)
    or vice-versa, so we can only change between Rock and
    Scissors. How do we calculate \(x\) and \(m\)? Here the
    Extended Euclidean Algorithm helps us - this simple code
    below allows us to calculate the values we are looking
    for in a matter of milliseconds:

    \begin{lstlisting}[language=Python]
def xgcd(b, a):
    x0, x1, y0, y1 = 1, 0, 0, 1
    while a != 0:
        q, b, a = b // a, a, b % a
        x0, x1 = x1, x0 - q * x1
        y0, y1 = y1, y0 - q * y1
    return  b, x0, y0
    \end{lstlisting}

    \noindent Running this as \verb|xgcd(10, 2**256)| will
    return three values - the greatest common divisor (2)
    and \(x\) and \(m\) we have been looking for. And here,
    we want to focus on \(x\), which is the `magic number'
    that allows us to perform the
    attack\footnote{\(x=92633671389852956338856788006950326282615987732512451231566067206330503711949\)}.
    Following the reasoning from above, we can now pick any
    two numbers \(a\) and \(b\) such that \(a - b = x\) and
    guarantee that \(10a = 10b + 2 \text{ (mod 2}^{256})\),
    which in turn allows us to always start the game by
    picking rock and nonce \(a\) and then, if we see that
    the opponent has revealed paper, change it to scissors
    and nonce \(b\). This way we win any time the opponent
    picks paper or scissors and draw whenever they pick
    rock, giving the desired \(66.(6)\%\) chance of a win
    and \(33.(3)\%\) change of a draw.
    \newline

    \noindent The script below produces a random pair \((a,
    b)\) described above to facilitate the attack:

    \begin{lstlisting}[language=Python]
import hashlib, random

magic = 92633671389852956338856788006950326282615987732512451231566067206330503711949

def attack():
    rand_int = random.randrange(0, magic)
    rock = magic + rand_int
    print("rock: {}".format(rock))
    rock = (rock * 10) % 2**256
    rock = rock.to_bytes(32, byteorder='big')
    rock_hash = hashlib.sha256(rock).hexdigest()
    scissors = rand_int
    print("scissors: {}".format(scissors))
    scissors = ((scissors * 10) + 2) % 2**256
    scissors = scissors.to_bytes(32, byteorder='big')
    scissors_hash = hashlib.sha256(scissors).hexdigest()

    # as we expect, those two will always be the same
    print("rock hash: 0x{}".format(str(rock_hash)))
    print("scis hash: 0x{}".format(str(scissors_hash)))

attack()
    \end{lstlisting}
    
    \subsubsection{Timestamp attack}
    Another very subtle type of attack requires one of the
    players to be a miner or to cooperate with one. As we
    know, in Ethereum the block timestamp tolerance is 900
    seconds (15 minutes) \cite{ethrules}. Therefore, a miner
    can post ones transaction on the block as if it happened
    15 minutes later than the current time.
    \newline

    \noindent Imagine the following situation. A miner is
    one of the players in the game and has the ability to
    reveal their hand first (this is an important part that
    will allow us to perform the attack). What the miner can
    do is reveal their value after the other player joins
    the game and then {\itshape immediately} call the
    \verb|finish| function in the code below and hope to
    mine the block with his call to the \verb|finish|
    function. If he manages to mine the block, he sets the
    block's timestamp to \verb|now + k| for \(k > 100\) and
    thus the function will be executed as if the time limit
    has passed and the other player did not reveal their
    choice, so the miner will get all the money for that
    game.

    \begin{lstlisting}[language=Solidity]
function reveal(string nonce, uint choice) public returns (string){
    /**
    ...
    */
    
    //record time of first reveal
    if (timeReveal==0) {
        timeReveal = now;
    }

    /**
    ...
    */
}

function finish() public {
    //both players need to have revealed or the second player needs to have missed the reveal window
    require ((hasRevealed[player1] && hasRevealed[player2]) || now-timeReveal > 100);
    
    //avoid re-entrancy
    uint payout = pot;
    pot = 0;
    
    //if one party did not reveal, pay the other party
    if (hasRevealed[player1] && !hasRevealed[player2]) {
        player1.transfer(payout);
        payout=0;
    }
    
    if (hasRevealed[player2] && !hasRevealed[player1]) {
        player2.transfer(payout);
        payout=0;
    }

    /**
    ...
    */
}
    \end{lstlisting}

    \noindent The obvious way to prevent this kind of attack
    would be to change the time check to a value bigger than
    900 seconds (e.g. 20 minutes = 1200 seconds) so that the
    miner cannot immediately call the \verb|finish| function
    and claim all the money as the check \verb|now - timeReveal > 1200| 
    won't pass anymore.
    \newline

    \noindent It is however worth noting that two popular
    Ethereum protocol implementations Geth and Parity both
    reject blocks with timestamps more than {\bfseries 15
    seconds} in the future \cite{ethsecu, geth, parity}.
    Therefore there exists a so-called 15-second rule that
    states that ``If the contract function can tolerate a
    15-second drift in time, it is safe to use
    block.timestamp'' \cite{ethsecu}, so this would most
    likely be safe anyway.

    \section{Game code} \label{s:code}
    Below you can find the Solidity code for my
    implementation of the rock-paper-scissors game.
    
    \begin{lstlisting}[language=Solidity]
pragma solidity ^0.4.16;

contract rpsContract {

    // This should be self-explanatory
    struct Player {
        address add;
        bool revealed;
        bool got_paid;
        uint256 hashed_choice;
        uint8 choice;
    }

    address owner;
    // Count if both players got paid (in terms of a tie)
    Player[2] players;
    // 0 - player 0, 1 - player 1, 2 - tie
    uint gameWinner;
    // Used to reset the game in case of inactivity
    uint256 timer;
    GamePhase gamePhase;
    uint256 gameStake;
    
    // Idle - waiting for new players
    // Started - one player started the game
    // Reveal - waiting for the players to reveal
    // Finished - ready to claim the Ether
    enum GamePhase { Idle, Started, Reveal, Finished }
    
    constructor () public {
        owner = msg.sender;
        gamePhase = GamePhase.Idle;
        gameStake = 0;
        players[0] = Player(0, false, false, 0, 0);
        players[1] = Player(0, false, false, 0, 0);
    }

    /**
    Start a new game or get into an existing one by sending a hashed (sha3)
    value of the choice and a random (chosen by the player) seed
     */
    function play(uint256 hashed_choice) public payable {
        require (gamePhase == GamePhase.Idle || gamePhase == GamePhase.Started, "Game is currently on");
        require (msg.value >= 1 ether, "Stakes need to be at least 1 ether");
        require (msg.value % 2 == 0, "Stakes need to be divisible by 2");

        if (gamePhase == GamePhase.Idle) { // first player starting the game
            // half of what you put in is treated as deposit
            // and half as the game prize pool
            gameStake = msg.value / 2;
            gamePhase = GamePhase.Started;
            players[0] = Player(msg.sender, false, false, hashed_choice, 0);
        } else { // second player joining the game
            require (msg.value / 2 == gameStake, "Stake needs to be equal to the other player's stake");
            gamePhase = GamePhase.Reveal;
            // start the timer to enable resetting the game
            // in case of inactivity
            timer = now;
            players[1] = Player(msg.sender, false, false, hashed_choice, 0);
        }
    }

    /**
    Reveal your choice by providing the nonce and choice
     */
    function reveal(uint256 nonce, uint8 choice) public {
        require(gamePhase == GamePhase.Reveal, "Game not in reveal phase yet");
        bytes memory reveal_val = abi.encodePacked(nonce, choice);
        uint8 p = determinePlayer(msg.sender);
        require(p != 2, "You are not taking part in the game");
        require(uint256(keccak256(reveal_val)) == players[p].hashed_choice, "Invalid seed and/or choice");

        // save the choice and mark that
        // the player have revealed the value
        players[p].choice = choice;
        players[p].revealed = true;

        if (players[0].revealed == true && players[1].revealed == true) {
            // both players revealed
            // determine winner and allow claiming winnings
            gamePhase = GamePhase.Finished;
            gameWinner = getWinner(int8(players[0].choice), int8(players[1].choice));
        } else {
            // reset the timer due to activity
            timer = now;
        }
    }

    /**
    Claim money or reset the game in case of inactivity
     */
    function claim() public {
        if (now - timer > 2 minutes && gamePhase == GamePhase.Reveal) {
            // prevent reentrancy attack
            reset();

            // 2 minutes have passed since last activity
            // allow anyone to reset the game
            if (players[0].revealed == true) {
                // first player revealed, gets all
                players[0].got_paid = true;
                players[0].add.transfer(4*gameStake);
            } else if (players[1].revealed == true) {
                // second player revealed, gets all
                players[1].got_paid = true;
                players[1].add.transfer(4*gameStake);
            } else {
                // no player revealed, split 50/50
                players[0].got_paid = true;
                players[1].got_paid = true;
                players[0].add.transfer(2*gameStake);
                players[1].add.transfer(2*gameStake);
            }

            return;
        }

        require(gamePhase == GamePhase.Finished, "Game not finished yet");
        uint8 p = determinePlayer(msg.sender);
        require(p != 2, "You are not taking part in the game");
        require(!players[p].got_paid, "You already have your money");

        if (gameWinner == p) {
            players[p].got_paid = true;
            // transfer 75% of total to winner
            msg.sender.transfer(3*gameStake);
        } else if (gameWinner != p && gameWinner != 2) {
            players[p].got_paid = true;
            // transfer 25% of total to loser
            msg.sender.transfer(gameStake);
        } else if (gameWinner == 2) {
            players[p].got_paid = true;
            // transfer each player 50% of total
            msg.sender.transfer(2*gameStake);
        }

        if (players[0].got_paid && players[1].got_paid) {
            reset();
        }
    }

    /**
    Determine the player based on the address.
    Returns 2 if player is unknown 
     */
    function determinePlayer(address add) private view returns(uint8) {
        if (players[0].add == add) {
            return 0;
        } else if (players[1].add == add) {
            return 1;
        } else {
            return 2;
        }
    }

    /**
    Reset the game state to idle to allow new games
    to be played
     */
    function reset() private {
        gamePhase = GamePhase.Idle;
    }
    
    /**
    Determine the winner given choices of both players
    0 + 3k means 'rock'
    1 + 3k means 'paper'
    2 + 3k means 'scissors'
     */
    function getWinner(int8 a, int8 b) public view returns (uint8) {
        require(gamePhase == GamePhase.Finished, "Game not finished yet");

        if ((a - b) % 3 == 1) {
            return 0;
        } else if ((a - b) % 3 == -1) {
            return 1;
        } else if ((a - b) % 3 == 2) {
            return 1;
        } else if ((a - b) % 3 == -2) {
            return 0;
        } else if ((a - b) % 3 == 0) {
            return 2;
        }
    }

    /**
    Helper function to be able to see what state
    the game is currently in
     */
    function getGamePhase() public view returns (string) {
        if (gamePhase == GamePhase.Idle) {
            return "Waiting for players";
        } else if (gamePhase == GamePhase.Started) {
            return "Waiting for player 2";
        } else if (gamePhase == GamePhase.Reveal) {
            return "Waiting for the players to reveal their choices";
        } else if (gamePhase == GamePhase.Finished) {
            if (gameWinner == 0) {
                return "Player 1 won, waiting to claim the prize";
            } else if (gameWinner == 1) {
                return "Player 2 won, waiting to claim the prize";
            } else {
                return "Tie, waiting for the players to claim the money";
            }
        }
    }

    /**
    Helper function to be able to see what is the
    stake you need to put in to compete
     */
    function getStake() public view returns (uint256) {
        return gameStake;
    }
} 
    \end{lstlisting}

    \noindent and the hash generating script in Python
    \begin{lstlisting}[language=Python]
from web3 import Web3
from random import randint

seed = randint(2**128, 2**256)
choice = int(input("Pick a choice: "))

h = Web3.soliditySha3(['uint256', 'uint8'], [seed, choice%3])
print("Seed-choice:", str(seed) + ", " + str(choice%3))
print("Hash:", int(h.hex(), 16))
    \end{lstlisting}

    \begin{thebibliography}{9}
        \bibitem{ypaper}
        G. Wood, Ethereum: A secure decentralised
        generalised transaction ledger,
        \url{https://gavwood.com/paper.pdf}

        \bibitem{ethrules} 
        Ethereum Block Protocol 2.0, wiki post created by
        Github users {\bfseries ruchevits} and {\bfseries
        heikoheiko},
        \url{https://github.com/ethereum/wiki/blob/c02254611f218f43cbb07517ca8e5d00fd6d6d75/Block-Protocol-2.0.md}
        
        \bibitem{ethsecu}
        Recommendations for Smart Contract Security in
        Solidity,
        \url{https://consensys.github.io/smart-contract-best-practices/recommendations/#timestamp-dependence}
        
        \bibitem{geth}
        Geth protocol implementation of the timestamp rule,
        \url{https://github.com/ethereum/go-ethereum/blob/4e474c74dc2ac1d26b339c32064d0bac98775e77/consensus/ethash/consensus.go#L45}

        \bibitem{parity}
        Parity protocol implementation of the timestamp
        rule,
        \url{https://github.com/paritytech/parity-ethereum/blob/73db5dda8c0109bb6bc1392624875078f973be14/ethcore/src/verification/verification.rs#L296-L307}
    \end{thebibliography}


\end{document}
