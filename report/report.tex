\documentclass{article}
\usepackage{amsfonts}
\usepackage{amssymb}
\usepackage{amsmath}
\usepackage{graphicx}
\usepackage{mathtools}
\usepackage{float}
\usepackage{bm}
\usepackage[margin=1in]{geometry}
\newcommand\bd{\bullet}
\newcommand\wt{\circ}
\makeatletter
% we use \prefix@<level> only if it is defined
\renewcommand{\@seccntformat}[1]{%
  \ifcsname prefix@#1\endcsname
    \csname prefix@#1\endcsname
  \else
    \csname the#1\endcsname\quad
  \fi}
% define \prefix@section
\newcommand\prefix@section[1]{\thesection. #1}
\newcommand\userule[1]{\mathrel{\overset{\makebox[0pt]{\mbox{\normalfont\tiny\sffamily #1}}}{=}}}
\makeatother
\title{BDL Coursework}
\author{s1554741}
\date{\today}

\begin{document}
    \maketitle

    \section{The King of Ether}
    The highest level overview of the smart contract for the
    King of Ether is that anyone can become the King by
    paying sufficient amount of Ether (at least as much as
    the last King).
    \\ \\
    When the contract is created, we are saving the
    information about the address of the owner in the
    \verb|owner| variable and then automatically the creator
    becomes the King with the message \verb|Let's play a game...| 
    and with the value 1 wei.
    \\ \\
    Then we have multiple functions that allow other parties
    to take part in the game and become the King themselves.
    \verb|claimThrone(string message)| function allows the
    user to pay a certain amount of Ether (\verb|payable|
    keyword) and if the value is greater or equal from the
    highest value (value that the last King payed), then the
    user becomes the King with the message they passed to
    the function. There is also a restriction on the amount
    of Ether one can use to become the King (50 Ether) that
    can be lifted by the owner of the contract as described
    later. Moreover, any time a new King is determined, the
    last King's earnings are saved in the \verb|earnings|
    map under his address. This way any user that was the
    King but got dethroned can recover their money using the
    \verb|withdraw()| function. 
    \\ \\
    We also have one getter in the form of
    \verb|getKingsTotal()| that allows to see how many kings
    were there in total over the lifespan of the contract
    and \verb|raiseRestriction()| that allows the owner to
    lift the restriction described two paragraphs above and
    allow users to pay more than 50 Ether to become the
    King.
    \\ \\
    Thankfully I managed to become the King at one point
    too. The ID of the transaction was 
    \[
        \texttt{0x903d0a4f95656cbfdbd520898fdfe811d39119e0d229a4effeb14a93d353a79f},
    \]
    my address is
    \[
        \texttt{0x47ADEE763A7BDE2a03c029725C5f7c9315f3B42a}
    \]
    and the message I used for the transaction was
    "\verb|Test transaction please ignore|".

    \section{Rock-paper-scissors}

\end{document}
